\documentclass[11pt]{article}
\usepackage{times}
\usepackage{setspace}
\singlespacing
\usepackage{amsmath,amssymb, amsthm}
\usepackage{graphicx}
\usepackage{bm}
\usepackage[hang, flushmargin]{footmisc}
\usepackage[colorlinks=true]{hyperref}
\usepackage[nameinlink]{cleveref}
\usepackage{footnotebackref}
\usepackage{url}
\usepackage{listings}
\usepackage[most]{tcolorbox}
\usepackage{inconsolata}
\usepackage[papersize={8.5in,11in}, margin=1in]{geometry}
\usepackage{float}
\usepackage{caption}
\usepackage{esint}
\usepackage{url}
\usepackage{enumitem}
\usepackage{subfig}
\usepackage{wasysym}
\newcommand{\ilc}{\texttt}
\newcommand{\p}{\partial}
\newcommand{\vphi}{\varphi}
\usepackage{etoolbox}
\usepackage{physics}
\usepackage{xcolor}
\usepackage{subfiles}
\patchcmd{\thebibliography}{\section*{\refname}}{}{}{}




% \makeatletter
% \renewcommand{\@seccntformat}[1]{}
% \makeatother

\begin{document}


\title{\textbf{CSDS 440 Class Project: \\Adversarial Machine Learning}}

\author{Shaochen (Henry) Zhong, \ilc{sxz517} \\ Minyang Tie, \ilc{mxt497} \\ Alex Useloff, \ilc{adu3} \\ Austin Keppers, \ilc{agk51} \\ David Meshnick, \ilc{dcm101}\\}
\date{Due and submitted on 12/04/2020 \\ Fall 2020, Dr. Ray}
\maketitle

\renewcommand{\abstractname}{Contributions}
\begin{abstract}
We as a group of five have looked into the field of adversarial machine learning and specifically investigated several algorithms regarding aspects of attack (evasion, poisoning) and defense (detector, transformer, pre-process). Inside the group, overall contribution are as following:
\begin{itemize}
    \item Henry:
    \begin{itemize}
        \item Read 3 papers on algorithms.
        \item Implemented 2 algorithms (\ilc{FGSM} and \ilc{Hop Skip Jump}) with 3 extensions.
        \item Piplined attack algorithms to work with 2 datasets, collected almost all (7 algorithms/useable extensions) attack experiments data (except \ilc{backdoor} and \ilc{one pixel attack}) for the comparative evaluations.
        \item Piplined and collected experiments data for \ilc{FGSM, Hop Skip Jump, DeepFool} attacks (and their useable extensions) against \ilc{Detector} and \ilc{Spatial Smoothing} defenses on 2 datasets.
        \item Implemented $L_2$ and $L_{\infty}$ \textit{perturbation budget} to aid comparative evaluation.
        \item Wrote \textbf{Introduction and Significance} section.
        \item Plotted all graphs and charts in \textbf{Comparative Study and Discussion}.
        \item Helped group move forward by making technical decisions, distributing works, setting up deadlines, and facilitating coordination between groupmates.
    \end{itemize}
    \item Minyang:
    \item Alex:
    \item Austin:
    \item David:
\end{itemize}
\end{abstract}



\setcounter{tocdepth}{4}
\vspace{0.5cm}
{\hypersetup{hidelinks}
\tableofcontents
}

\newpage

\section{Introduction and Significance}

\section{Individual Reports}

\subsection{Shaochen (Henry) Zhong's Individual Report}
\subsubsection{Overview}


\subsubsection{\ilc{Fast Gradient Sign Method}}
\paragraph{Algorithm Intuition}
\paragraph{Algorithm Implementation}
\paragraph{Experiments}
\paragraph{Evaluation}

\subsubsection{\ilc{Hop Skip Jump}}
\paragraph{Algorithm Intuition}
\paragraph{Algorithm Implementation}
\paragraph{Experiments}
\paragraph{Evaluation}


\subsubsection{\ilc{Feature Collision}}
\paragraph{Algorithm Intuition}


\subsection{Minyang Tie's Individual Report}


\subsection{Alex Useloff's Individual Report}


\subsection{Austin Keppers' Individual Report}
\subfile{individual_reports/agk51}

\subsection{David Meshnick's Individual Report}

\section{Comparative Study and Discussion}

\subsection{Overview}
\subsection{Attack Algorithms}
\subsubsection{Evasion}
\subsubsection{Poisoning}
\subsection{Defense Algorithms}




% % % % % % % % % % % % % % % % % % % % % % % % % % % % % % % % % %
% % % % % % % % % % % % % % % % % % % % % % % % % % % % % % % % % %
% % % % % % % % % % % % % % % % % % % % % % % % % % % % % % % % % %
% \section{References}
% \nocite{*}
% \raggedright
% \bibliography{references.bib}
% \bibliographystyle{plain}




\end{document}